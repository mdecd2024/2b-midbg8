\renewcommand{\baselinestretch}{1.5} %設定行距
\pagenumbering{roman} %設定頁數為羅馬數字
\clearpage  %設定頁數開始編譯
\sectionef
\addcontentsline{toc}{chapter}{摘~~~要} %將摘要加入目錄
\begin{center}
\LARGE\textbf{摘~~要}\\
\end{center}
\begin{flushleft}
\fontsize{14pt}{20pt}\sectionef\hspace{12pt}\quad 本學期採取個人及團體分組來學習,團體實習目標為開發一款能在 web-based CoppeliaSim 場景中雙方或多方對玩的遊戲。pj3 為八人一組,根據自選產品在期限內完成產品開發,在 w16 現場發表八人協同四週後所完成的產品,在 w17 各組採 OBS + Teams 以影片發表所完成的協同產品。\\[12pt]
\fontsize{14pt}{20pt}\sectionef\hspace{12pt}\quad  接續 pj2,各組須對雙輪車進行設計改良,以提升行進與對戰效率。採 CAD 進行場景與多輪車零組件設計後,轉入足球場景中以鍵盤 arrow keys 與 wasd 等按鍵進行控制,對雙方每組將有四名輪車球員,且每兩人在同一台電腦上操作,完成後各組須在分組網站中提供所有相關檔案下載連結,且提供線上分組簡報與分組 pdf 報告連結。\\[12pt]	
\fontsize{14pt}{20pt}\sectionef\hspace{12pt}\quad 此專題是雙方利用各四台 BubbleRob 多輪車在一足球場景中進行對戰,雙方球門分別設有感測器。 在規定時間內,每進一球即透過程式重新往球場內隨機發球,接續賽局。模擬場景中還須配置 LED 計分板顯示比賽剩餘時間與比分,還需另外建立以機械轉盤傳動計分系統。在 CoppeliaSim 模擬環境中進行測試運用上的可行性並嘗試透過埠號供使用者觀看。\\
  更多詳細內容可以到 https://mdecd2023.github.io/2a3-pj3ag4 了解。\\[10pt]

\end{flushleft}
\newpage
%=--------------------Abstract----------------------=%
\renewcommand{\baselinestretch}{1.5} %設定行距
\addcontentsline{toc}{chapter}{Abstract} %將摘要加入目錄
\begin{center}
\LARGE\textbf\sectionef{Abstract}\\
\begin{flushleft}
\fontsize{14pt}{16pt}\sectionef\hspace{12pt}\quad This semester adopts both individual and group learning approaches. The group project involves developing a web-based game that allows two or more players to interact in a CoppeliaSim simulation environment. For Project 3 (pj3), the group consists of eight members who will collaborate to develop a product of their choice within a given deadline. In Week 16, the group will present the product they have developed over four weeks of collaboration. In Week 17, each group will use OBS + Teams to present their collaborative product through a video presentation.\\[12pt]
\fontsize{14pt}{16pt}\sectionef\hspace{12pt}\quad Continuing from pj2, each group is required to design improvements for a two-wheeled vehicle to enhance its mobility and combat efficiency. Using CAD software, the groups will design the scene and components of the multi-wheel vehicle. The project will then transition to a soccer field scenario, where the vehicle will be controlled using keyboard arrow keys and WASD keys. Each group will have four vehicle players, with two players operating on the same computer. Upon completion, each group must provide download links for all relevant files on the group's website, as well as links to online group presentations and a PDF report.\\[12pt]
\fontsize{14pt}{16pt}\sectionef\hspace{12pt}\quad This project involves a two-player battle using four BubbleRob multi-wheel vehicles in a soccer field scenario. Each player has their own goal equipped with sensors. Within a specified time frame, each goal scored triggers the program to reset and randomly kick off a new ball into the field, continuing the game. The simulation scene also includes an LED scoreboard to display the remaining time and score of the match. Additionally, a mechanical turntable-driven scoring system needs to be created. Feasibility testing and user observation will be conducted in the CoppeliaSim simulation environment, with the option for users to observe through port numbers.\\[12pt]
\end{flushleft}
